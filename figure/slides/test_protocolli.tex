\newcommand{\smallqueueS}[2]{%
\draw[queuesty] (#1)node[right,xshift=.2cm,yshift=.-0.4cm]{\sffamily #2}-- ++(.2,.15)-- ++(.9,0)-- ++(0,-.3)-- ++(-.9,0)-- cycle;
\draw[fill=green!80] (#1) ++ (.3,.15)rectangle +(.3,-.3)node[font=\sffamily\tiny,inner sep=0pt,outer sep=-2pt, midway]{$L_1$};
\draw[fill=orange!80] (#1) ++ (.7,.15)rectangle +(.3,-.3)node[font=\sffamily\tiny,inner sep=0pt,outer sep=-2pt, midway]{$L_3$};}
\newcommand{\bigqueueS}[2]{%
\draw[queuesty] (#1)node[right,xshift=.2cm]{\sffamily #2}-- ++(.3,.3)-- ++(.8,0)-- ++(0,-.6)-- ++(-.8,0)-- cycle;}

\newcommand{\exampleTest}[2]{%
\begin{tikzpicture}[
  xscale=#1,
  yscale=#2,
  every node/.append style={transform shape},
  queuesty/.style={fill=white, very thick, font=\tiny},
  cpusty/.style={fill=gray!40, draw, circle, minimum width=1cm},
  srpsty/.style={fill=white, draw, circle, text width=.17cm, font=\tiny, very thick},
  numsty/.style={text width=.1cm, font=\tiny},
  ressty/.style={fill=red!30, draw, very thick, rounded corners=5pt},
  arrow/.style={->,>=stealth},
  littletext/.style={font=\sffamily\tiny,inner sep=0pt,outer sep=-2pt,fill=white},
  emptytask/.style={rectangle, minimum width=.7cm,font=\footnotesize},
  taskA/.style={fill=green!40, draw, rectangle, minimum width=.7cm,font=\footnotesize},
  taskB/.style={fill=orange!40!yellow, draw, rectangle, minimum width=.7cm,font=\footnotesize},
  task/.style={fill=white, draw, rectangle, minimum width=.7cm,font=\footnotesize}]

\begin{scope}[xshift=4cm, yshift=1cm]
\coordinate (SRP1node) at (0,0);
\node[cpusty] (P1) at (2.5,0) {$P_1$};
\node[taskA] (T1) at (1.3,0.8) {$L_1$};
\node[emptytask] (TP1) at (1.3,0.4) {$\cdots$};
\node[task] (TJ) at (1.3,0) {$J_i$};
\node[emptytask] (TP2) at (1.3,-0.4) {$\cdots$};
\node[taskA] (T2) at (1.3,-0.8) {$H_2$};
\draw[arrow] (T1.east) -- (P1.west);
\draw[arrow] (TJ.east) -- (P1.west);
\draw[arrow] (T2.east) -- (P1.west);
\draw[dashed, thin] ([shift={(-.1,.2)}]T1.north-|SRP1node) node[right,xshift=.3cm,littletext]{partition$_1$} rectangle ([shift={(.1,-.1)}]T2.south-|P1.east);
\node[srpsty] (SRP1) at (SRP1node) {}; \node[font=\sffamily\tiny] at(SRP1node.east){SRP};
\draw[arrow] (T1.west) to[out=180,in=0] ([yshift=.1cm]SRP1.east);
\node[numsty] (N1) at (0.50,.65) {}; \node[font=\sffamily\tiny] at(N1){\color{red}1};


\node[numsty] (N3) at (1.9,-.45) {}; \node[font=\sffamily\tiny] at(N3){\color{red}3};

\end{scope}

\begin{scope}[xshift=4cm, yshift=-1.1cm]
\coordinate (SRP2node) at (0,-.8);
\node[cpusty] (P2) at (2.5,-0.8) {$P_2$};

\node[task] (TX) at (1.3,0) {$J_x$};
\node[emptytask] (TP3) at (1.3,-0.4) {$\cdots$};
\node[taskB] (T3) at (1.3,-0.8) {$L_3$};
\node[emptytask] (TP4) at (1.3,-1.2) {$\cdots$};
\node[task] (TY) at (1.3,-1.6) {$J_y$};

\draw[arrow] (T3.east) -- (P2.west);
\draw[arrow] (TX.east) -- (P2.west);
\draw[arrow] (TY.east) -- (P2.west);

\draw[dashed, thin] ([shift={(-.1,.2)}]TX.north-|SRP2node) node[right,xshift=.3cm,littletext]{partition$_2$} rectangle ([shift={(.1,-.1)}]TY.south-|P2.east);
\node[srpsty] (SRP2) at (SRP2node) {}; \node[font=\sffamily\tiny] at(SRP2node.east){SRP};
\draw[arrow] (T3.west) to[out=180,in=0] (SRP2.east);
\node[numsty] (N2) at (0.6,-0.6) {}; \node[font=\sffamily\tiny] at(N2){\color{red}2};


\end{scope}

\begin{scope}
\draw[ressty] (-0.5,-.65) rectangle +(1.5,.5) node[midway, font=\tiny]{resource};
\smallqueueS{1.0,-0.4}{FIFO}
\draw[|-|] (1.2, -.02) -- ++(.9,0)node[midway,fill=white,font=\tiny]{$2$};
\end{scope}

\draw[arrow] (SRP1.west)node[anchor=south east,yshift=-.3cm,font=\tiny\sffamily]{ \begin{tabular}{c} top of access queue, \\ acquires resource \end{tabular}} to[out=180,in=0] (2.1,-.3);
\draw[arrow] (SRP2.west)node[anchor=north east,yshift=.3cm,font=\tiny\sffamily]{ \begin{tabular}{c} not top of access queue, \\ spins locally \end{tabular}} to[out=180,in=0] (2.1,-.5);

\end{tikzpicture}}
