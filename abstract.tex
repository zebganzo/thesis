\textit{Multiprocess Resource Sharing Protocol} propone un approccio innovativo per la condivisione di risorse globali. Burns e Wellings delineano una variante multiprocessor di \textit{Priority Ceiling Protocol} con l'obiettivo di utilizzare, per sistemi multiprocessor partizionati, le tecniche di analisi di schedulabilità dei sistemi single processor. Per poterlo fare, occorre che il tempo di attesa per accedere alle risorse rifletta la contesa parallela, dovuta alla condivisione tra più processori, ma limitando il tempo di attesa dei job in coda senza precludere l'indipendenza dei job a priorità superiore che non la richiedono. MrsP prevede che l'esecuzione della sezione critica corrispondente alla risorsa, in caso di prerilascio del suo possessore, possa essere proseguita da parte del primo job della coda in attesa di accederla. In questa tesi, dopo aver analizzato nel dettaglio il protocollo, è elaborata una proposta di soluzione in grado di gestire anche gli aspetti che, nell'ambito teorico, non sono considerati. L'implementazione fornita è valutata tramite esperimenti che mirano a calcolare il costo delle singole primitive, valutare i costi aggiunti dal protocollo e, infine, confrontare MrsP con altri approcci.